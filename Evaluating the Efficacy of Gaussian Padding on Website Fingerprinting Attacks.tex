\documentclass[
	ruledheaders=chapter,
	class=report,
	thesis={type=master, department=inf},
	accentcolor=1c,
	custommargins=true,
	marginpar=false,
	parskip=half-,
	fontsize=11pt,
]{tudapub}

\usepackage[ngerman,main=english]{babel}

\usepackage[autostyle]{csquotes}
\usepackage{microtype}

\usepackage{enumerate}
\usepackage{enumitem}

\usepackage{biblatex}
\bibliography{thesis-literature}

\usepackage{tabularx}
\usepackage{booktabs}

\usepackage{mathtools}

\usepackage{graphicx}
\usepackage{float}
\usepackage[export]{adjustbox}
\usepackage[font=small,labelfont=bf]{caption}
\graphicspath{ {./images/} }

\counterwithout{footnote}{chapter}

\pagenumbering{roman}

\begin{document}
	\Metadata{
		title=Evaluating the Efficacy of Gaussian Padding on Website Fingerprinting Attacks,
		author=Johannes Leupold
	}
	
	\title{Evaluating the Efficacy of Gaussian Padding on Website Fingerprinting Attacks}
	\author[J. Leupold]{Johannes Leupold}
	\birthplace{Dresden}
	\reviewer{Jean-Paul Degabriele \and Some Other Guy}

	\department{inf}
	\institute{IT Security}
	\group{Cryptography and Network Security}
	
	\submissiondate{01.09.2021}
	\examdate{20.09.2021}
	
	% Hinweis zur Lizenz:
	% TUDa-CI verwendet momentan die Lizenz CC BY-NC-ND 2.0 DE als Voreinstellung.
	% Die TU Darmstadt hat jedoch die Empfehlung von dieser auf die liberalere
	% CC BY 4.0 geändert. Diese erlaubt eine Verwendung bearbeiteter Versionen und
	% die kommerzielle Nutzung.
	% TUDa-CI wird im nächsten größeren Release ebenfalls diese Anpassung vornehmen.
	% Aus diesem Grund wird empfohlen die Lizenz manuell auszuwählen.
	% \tuprints{urn=1234,printid=12345,doi=10.25534/tuprints-1234,license=cc-by-4.0}
	% To see furhter information on the license option in English, remove the license= key and pay attention to the warning & help message.
	
	% \dedication{Für alle, die \TeX{} nutzen.}
	
	\maketitle
	
	\affidavit[digital]% oder \affidavit[digital] falls eine rein digitale Abgabe vorgesehen ist.

	\begin{abstract}[english]
		Abstract
	\end{abstract}
	
	\tableofcontents
	
	\chapter{Introduction}
	\label{introduction}
	\IMRADlabel{introduction}
	\pagenumbering{arabic}

	This is the introduction.

	\chapter{Background Material}
	\label{fingerprinting}

	\section{Theoretical Setting}
	\label{theoretical}
	
	\subsection{Threat Model}
	\label{theoretical:threat_model}
	
	\subsection{Attacks and Defenses}
	\label{theoretical:attacks_and_defenses}
	
	\section{Gaussian Padding}
	\label{gaussian_padding}
	
	\section{Security Bound Estimation according to Cherubin \cite{Cherubin2017}}
	\label{cherubin_bounds}

	\chapter{Prior Work}
	\label{related}

	\chapter{Experimental Methodology}
	\label{methods}
	\IMRADlabel{methods}

	\section{Trace Data}
	\label{dataset_quality}

	\section{Evaluating Attack Performance}
	\label{pipeline}

	\section{Error Bound Estimation}
	\label{error_bound_estimation}

	\chapter{Results}
	\label{results}
	\IMRADlabel{results}

	\section{Empirical Performance of Gaussian Padding}
	\label{performance}

	\section{Estimated Security Bounds}
	\label{security_bounds}

	\chapter{Discussion}
	\label{discussion}
	\IMRADlabel{discussion}

	\chapter{Conclusion}
	\label{conclusion}

	\pagebreak
	\pagenumbering{roman}
	\setcounter{page}{5}
	\addcontentsline{toc}{chapter}{Bibliography}
	\printbibliography

	\cleardoublepage
	\phantomsection
	\addcontentsline{toc}{chapter}{\listfigurename}
	\listoffigures

	\cleardoublepage
	\phantomsection
	\addcontentsline{toc}{chapter}{\listtablename}
	\listoftables
\end{document}